\documentclass[a4paper]{article}
    \usepackage{fullpage}
    \usepackage{amsmath}
    \usepackage{amssymb}
    \usepackage{textcomp}
    \usepackage{hyperref}
    \usepackage[utf8]{inputenc}
    \usepackage[T1]{fontenc}
    \textheight=10in
    \pagestyle{empty}
    \raggedright

    %\renewcommand{\encodingdefault}{cg}
%\renewcommand{\rmdefault}{lgrcmr}

\def\bull{\vrule height 0.8ex width .7ex depth -.1ex }

% DEFINITIONS FOR RESUME %%%%%%%%%%%%%%%%%%%%%%%

\newcommand{\area} [2] {
    \vspace*{-9pt}
    \begin{verse}
        \textbf{#1}   #2
    \end{verse}
}

\newcommand{\lineunder} {
    \vspace*{-8pt} \\
    \hspace*{-18pt} \hrulefill \\
}

\newcommand{\header} [1] {
    {\hspace*{-18pt}\vspace*{6pt} \textsc{#1}}
    \vspace*{-6pt} \lineunder
}

\newcommand{\employer} [3] {
    { \textbf{#1} (#2)\\ \underline{\textbf{\emph{#3}}}\\  }
}

\newcommand{\contact} [3] {
    \vspace*{-10pt}
    \begin{center}
        {\Huge \scshape {#1}}\\
        #2 \\ #3
    \end{center}
    \vspace*{-8pt}
}

\newenvironment{achievements}{
    \begin{list}
        {$\bullet$}{\topsep 0pt \itemsep -2pt}}{\vspace*{4pt}
    \end{list}
}

\newcommand{\schoolwithcourses} [4] {
    \textbf{#1} #2 $\bullet$ #3\\
    #4 \\
    \vspace*{5pt}
}

\newcommand{\school} [4] {
    \textbf{#1} #2 $\bullet$ #3\\
    #4 \\
}
% END RESUME DEFINITIONS %%%%%%%%%%%%%%%%%%%%%%%

    \begin{document}
\vspace*{-40pt}

    

%==== Profile ====%
\vspace*{-10pt}
\begin{center}
    {\Huge \scshape {Bhavya Bhatt}}\\
    Mandi, Himachal Pradesh $\cdot$ bhavyabhatt17@gmail.com $\cdot$ +91 8219119315\\
\end{center}

%==== Education ====%
\header{EDUCATION}
\textbf{Indian Institute of Technology, Mandi}\hfill Mandi, Himachal Pradesh\\
B. Tech., Computer Science (Aug 2016 - July 2020)\\
 \textit{GPA: 8.07 (Upto 6th Semester)}
\vspace{2mm}

\header{RELEVANT COURSEWORK TAKEN UP}
\textbf{Fourth Semester: }
\begin{itemize} \itemsep 1pt
    \item Mechanics of Particles and Waves - General introduction to Lagrangian and Hamiltonian Mechanics\\
    \item Classical Electrodynamics\\
    \item Continuum Mechanics - Advanced fluid dynamics, general tensor formalism, Naiver-Stokes equation, energy conditions, linear and nonlinear fluids, numerical methods for solving velocity potential field under given boundary conditions\\
    \item Special Topics in High Energy Physics - Dirac Equation, Feynman diagrams, particle interactions and calculations of invariant amplitude (electron-positron scattering or electron-meson interaction), Quantum Electrodynamics \\
\end{itemize}
\textbf{Fifth Semester: }
\begin{itemize} \itemsep 1pt
    \item Special Topics in Quantum Mechanics - Non-relativistic and relativistic scattering theory, second quantization and related formalism, angular momentum theory and spin formalism\\
\end{itemize}
\textbf{Sixth Semester: }
\begin{itemize} \itemsep 1pt
    \item Advanced Statistical Mechanics\\
\end{itemize}
\vspace{2mm}

\header{AREA OF INTEREST}
\begin{itemize}
    \item Quantum Field Theory: gauge theories and their generalization to curved spacetime. Formulation of emergent classical spacetime from quantized spacetime.
    \item General Relativity: modifications in Einstein's gravity involving torsion.
    \item Cosmology
    \item Mathematics: Analysis on manifolds, topology, differential geometry, group theory, stochastic processes and stochastic calculus, chaos theory.
    \item Computational Physics: Implementations of optimized numerical algorithms relevant in the field of astrophysics. 
    \item Learning Theory: Theoretical statistical computational learning theory.
    
\end{itemize}

\header{PUBLICATIONS}
\textbf{New approach to path integral formulation of collapse models - "Resolution of Quantum Measurement Problem in non-relativistic case"} \\

{Summer Research Intern - June 2018 - August 2018} \\
\begin{itemize}
\item Proposed a new approach for path integral formulation of collapse models like GRW and other "all particle dynamics" theories. The work resulted into a paper "Path integrals, spontaneous localization and classical limit". \url{https://arxiv.org/abs/1808.04178}.
\end{itemize}
\newpage
\header{PROJECTS}
\textbf{Second-Order phase transitions in neural based learning models: }\\
\begin{itemize}
    \item This project is a sub part of my major technical project at IIT Mandi. This project deals with theoretical studies of learning algorithms using neural network models and their bifurcation limits.
    \item Current neural based models assumes only first-order linear dependence between the attributes of data and impose non-linearity on these first-order terms.
    \item The whole formalism shatters when there is significant second-order dependence which can have critical phase transitive behaviour in gradient field which in turn results in large variations across batches of data.
    \item This large variations results in just addition of random noise to the parameters of the model and affects learning of the model significantly.
    \item This project tries to formalize a new framework for second-order learning in which we can make parameters of the models as statistical fields and study their critical phase transition exponents and bifurcation limits.
\end{itemize}
\vspace*{2mm}
\textbf{Path Integrals formulation for Collapse Models: }\\
\begin{itemize}
    \item For my summer research internship at TIFR (Tata Institute of Fundamental Research, Mumbai), I worked under the guidance of Dr. Tejinder Pal Singh (Senior Prof. - Dept. Of Astronomy and Astrophysics), leading a group of 5 students from various IITs.
    \item As part of the project, I had the opportunity to formulate path integral approach for some of the collapse models (mainly GRW, QMUPL and CSL models) of quantum measurement problems.
    \item While working on the formulation, I was able to find new approaches for the above stated problem (which was conventionally done by comparing the noise function and the imaginary potential in the action for the propagator) by the proper application of jump operators in every infinitesimal time interval, with appropriate probabilities (Poisson process) and also by calculating the final density matrix function(since the probabilistic model involved mixed states).
    \item We challenged the idea that the classical limit of quantum mechanics is not just Planck constant tends to zero, but also some mechanism to kill superposition (which is the reason why we don’t observe superposition in macroscopic world) which has candidate theory such as collapse models and only then we can recover classical statistical limit like Liouville equation or Hamilton-Jacobi equation.
\end{itemize}
\vspace*{2mm}
\textbf{EinsteinPy: a Python package for Numerical Relativity} \\
\begin{itemize}
    \item This package was founder by me and my enthusiastic batch mates who were struggling to learn numerical relativity but was not able to find any software support for beginners.
    \item This library is first to provide support for numerical relativity and relativistic astrophysics problems in Python programming language.
    \item EPY provides a clean interface for code implementation which can be used by anyone who has little or no programming background and want to simulate their relativistic systems.
    \item I am physics advisor and non-core developer in the organisation.
\end{itemize}
\vspace*{2mm}
\newpage
\textbf{PyGlow: a Python package for Information Theory of Deep Learning} \\
\begin{itemize}
    \item I am the author of this package and is part of an ongoing final year major technical project in the field of ”Mathematics of Deep Learning”. The Project aims at developing new theoretical ideas which can provide mathematically formal answers to some of the profound questions in the field of deep learning.
    \item These questions include the mysteries of generalization, optimal architectures, memorization and compression phase in context of deep neural networks.
    \item The project demands the need for exploring cross field topics from information theory, statistical physics, group theory and complexity theory and experiment with these ideas in code.
    \item As a result of this project, all the experimentation code is available in form of a Python library package PyGlow which can be installed from PyPI with command ”pip install PyGlow”.
    \item This library is also one of the attempts to develop keras like API in PyTorch backend.
\end{itemize}
\vspace*{2mm}

\textbf{Non-Geodesic Raychaudhuri Equation: }\\
Having always been fascinated by general relativity, I am currently working on an interesting and self-thought out project on formalising the stress vector being applied on cosmological fluid( fluid mechanics in Riemannian and Pseudo Riemannian geometry) and to study the motion of the resultant non-geodesic curves. It also include the formalism of the fracture point of the material, mainly using the B tensor, it’s decomposition and Raychaudhuri equation, for non-geodesic congruences. \\
\vspace{2mm}
\newpage

\header{AWARDS AND ACHIEVEMENTS}
\begin{itemize} \itemsep 1pt
    \item Coauthor of a scientific paper submitted in the Physical Review Journal titled, “Path integrals, spontaneous localization and classical limit"\\
    link to the paper on arXiv - \href{https://bit.ly/2AOwPsf}{Path integrals, spontaneous localisation, and classical limit}.
    \item Speaker at Space Technology and Astronomy Cell(STAC), IIT Mandi - Astronomy and Astrophysics Department on the topic “General Relativity and Geometry in Physics” for the academic year 2017 - 2018 \\
    \item Secured 1st position in TopCoder Hackathon for Euler’s Notes - Android based project for hearing imparied people \\
\end{itemize}
\vspace{2mm}

\header{SKILLS}
\textbf{Computer Science and Applied Mathematics: }\\
Data analysis, data Mining and machine learning algorithms, deep learning and related optimization mathematics, parallel computing platforms (Nvidia CUDA, openmpi) and basic parallel algorithms, basic knowledge of quantum computing. \\
\vspace*{2mm}
\textbf{Other Engineering Skills: } \\
Computer programming in Python, C, C++, C\#, Fortran, on single and multi-core machine, distributed and parallel platforms, signal processing, probability and random processes, digital electronics. \\
\vspace*{2mm}
\textbf{Software skills: }\\
Have proficient knowledge in various software tools like:
\begin{itemize} \itemsep 1pt
    \item PyTorch (Advanced) - an open-source software library supporting Tensors and Dynamic neural networks in Python with strong GPU acceleration \\
    \item Scikit-learn - a free software machine learning library for the Python programming language \\
    \item  Keras - a high-level API to build and train deep learning models \\
    \item TensorFlow - an open-source software library for dataflow programming across a range of tasks \\
    \item SymPy - a Python library for symbolic computation \\
    \item Wolfram Mathematica - a modern technical computing system spanning most areas of technical computing \\
    \item Einstein Toolkit - a community-driven software platform of core computational tools to advance and support research in relativistic astrophysics \\
\end{itemize}
\vspace{2mm}

\header{HOBBIES}
\begin{itemize}
\item Sketching
\end{itemize}
\vspace{2mm}
\ 
\end{document}
